\documentclass[answers]{exam}
\usepackage[spanish]{babel}
\usepackage{amsthm,amssymb,amsfonts,amsmath}
\usepackage{enumitem}

\title{Lista 3 de \'Algebra Lineal}
\author{
	Castro Garc\'ia Jos\'e Luis\\
	Mac\'ias Hern\'andez V\'ictor Hugo
}
\makeatletter
\renewcommand*\env@matrix[1][*\c@MaxMatrixCols c]{%
   \hskip -\arraycolsep
   \let\@ifnextchar\new@ifnextchar
   \array{#1}}
\makeatother

\renewcommand{\solutiontitle}{\noindent\textbf{Soluci\'on}\par\noindent}

\begin{document}
\maketitle
\begin{questions}
	% PREGUNTA 1
	\question Sean $V=\mathbb{R}^4$ y $\mathbb{K}=\mathbb{R}$. Pruebe que $V$ es un espacio vectorial sobre $\mathbb{K}$ con la adici\'on y el producto por escalares usuales.
	\begin{solution} 
		Sea $\vec{u}=(u_1,u_2,u_3,u_4),\vec{v}=(v_1,v_2,v_3,v_4),\vec{w}=(w_1,w_2,w_3,w_4)\in V$ y $\alpha,\beta\in \mathbb{K}$.
		\begin{enumerate}[label=\roman*)]
			\item $
				\vec{u}+(\vec{w}+\vec{v})
				=(u_1,u_2,u_3,u_4)+((v_1,v_2,v_3,v_4)+(w_1,w_2,w_3,w_4))
				=(u_1,u_2,u_3,u_4)+(v_1+w_1,v_2+w_2,v_3+w_3,v_4+w_4)
				=(u_1+(v_1+w_1),u_2+(v_2+w_2),u_3+(v_3+w_3),u_4+(v_4+w_4))
				=((u_1+v_1)+w_1,(u_2+v_2)+w_2,(u_3+v_3)+w_3,(u_4+v_4)+w_4)
				=(u_1+v_1,u_2+v_2,u_3+v_3,u_4+v_4)+(w_1,w_2,w_3,w_4)
				=((u_1,u_2,u_3,u_4)+(v_1,v_2,v_3,v_4))+(w_1,w_2,w_3,w_4)
				=(\vec{u}+\vec{v})+\vec{w}
				$
			\item $
				\vec{u}+\vec{v}
				=(u_1,u_2,u_3,u_4)+(v_1,v_2,v_3,v_4)
				=(u_1+v_1,u_2+v_2,u_3+v_3,u_4+v_4)
				=(v_1+u_1,v_2+u_2,v_3+u_3,v_4+u_4)
				=(v_1,v_2,v_3,v_4)+(u_1,u_2,u_3,u_4)
				=\vec{v}+\vec{u}
				$
			\item Sea $0_v=(0,0,0,0)$. Luego\\
				$0_v+\vec{u}
				=(0,0,0,0)+(u_1,u_2,u_3,u_4)
				=(0+u_1,0+u_2,0+u_3,0+u_4)
				=(u_1,u_2,u_3,u_4)
				=\vec{u}
				$
			\item Sea $-\vec{u}=(-u_1,-u_2,-u_3,-u_4)$. Luego\\
				$\vec{u}+(-\vec{u})
				=(u_1,u_2,u_3,u_4)+(-u_1,-u_2,-u_3,-u_4)
				=(u_1+(-u_1),u_2+(-u_2),u_3+(-u_3),u_4+(-u_4))
				=(0,0,0,0)
				=0_v
				$
			\item $
				\alpha (\beta \vec{u})
				=\alpha (\beta (u_1,u_2,u_3,u_4))
				=\alpha (\beta u_1,\beta u_2,\beta u_3,\beta u_4)
				=(\alpha (\beta u_1),\alpha (\beta u_2),\alpha (\beta u_3),\alpha (\beta u_4))
				=((\alpha \beta) u_1,(\alpha \beta) u_2,(\alpha \beta) u_3,(\alpha \beta) u_4)
				=(\alpha \beta) (u_1,u_2,u_3,u_4)
				=(\alpha \beta) \vec{u}
				$
			\item $
				\alpha (\vec{u}+\vec{v})
				=\alpha ((u_1,u_2,u_3,u_4)+(v_1,v_2,v_3,v_4))
				=\alpha (u_1+v_1,u_2+v_2,u_3+v_3,u_4+v_4)
				=(\alpha (u_1+v_1),\alpha (u_2+v_2),\alpha (u_3+v_3),\alpha (u_4+v_4))
				=(\alpha u_1+\alpha v_1,\alpha u_2+\alpha v_2,\alpha u_3+\alpha v_3,\alpha u_4+\alpha v_4)
				=(\alpha u_1,\alpha u_2,\alpha u_3,\alpha u_4)+(\alpha v_1,\alpha v_2,\alpha v_3,\alpha v_4)
				=\alpha (u_1,u_2,u_3,u_4)+\alpha (v_1,v_2,v_3,v_4)
				=\alpha \vec{u}+\alpha \vec{v}
				$
			\item $
				(\alpha + \beta)\vec{u}
				=(\alpha + \beta)(u_1,u_2,u_3,u_4)
				=((\alpha + \beta)u_1,(\alpha + \beta)u_2,(\alpha + \beta)u_3,(\alpha + \beta)u_4)\\
				=(\alpha u_1 + \beta u_1,\alpha u_2 + \beta u_2,\alpha u_3 + \beta u_3,\alpha u_4 + \beta u_4)
				=(\alpha u_1,\alpha u_2,\alpha u_3,\alpha u_4)+(\beta u_1,\beta u_2,\beta u_3,\beta u_4)
				=\alpha (u_1,u_2,u_3,u_4)+\beta (u_1,u_2,u_3,u_4)
				=\alpha \vec{u}+\beta \vec{u}
				$
			\item $
				1 \cdot \vec{u}
				=1 \cdot (u_1,u_2,u_3,u_4)
				=(1 \cdot u_1,1 \cdot u_2,1 \cdot u_3,1 \cdot u_4)
				=(u_1,u_2,u_3,u_4)
				=\vec{u}
				$
		\end{enumerate}
	\end{solution}

	% PREGUNTA 2
	\question Sea $X$ un conjunto no vac\'io y $\mathbb{K}$ un campo. Sea $V=\{f : X \to \mathbb{K}\}$ es decir $V$ es el conjunto de todas las funciones que van de $X$ a $\mathbb{K}$. Para todas $f,g\in V$ y para toda $\alpha \in \mathbb{K}$ definimos $f+g$ y $\alpha f$ como $(f+g)(x)=f(x)+g(x)$ y $(\alpha f)(x)=\alpha f(x)$ para toda $x\in X$. Pruebe que $V$ es un espacio vectorial sobre $\mathbb{K}$.
	\begin{solution}
		Sea $f,g,h\in V$ y $\alpha,\beta\in \mathbb{K}$. Ademas resulta que $\forall x\in X$; $f(x),g(x),h(x)\in \mathbb{K}$.
		\begin{enumerate}[label=\roman*)]
			\item $f+(g+h)=(f+(g+h))(x)=f(x)+(g+h)(x)=f(x)+g(x)+h(x)=(f+g)(x)+h(x)=((f+g)+h)(x)=(f+g)+h$
			\item $f+g=(f+g)(x)=f(x)+g(x)=g(x)+f(x)=(g+f)(x)=g+f$
			\item Definiremos a $0_v$ como la funci\'on $0_f:X \to K$ tal que $\forall x\in X$ resulte que $0_f(x)=0_k$, luego:\\
			$f+0_f=(f+0_f)(x)=f(x)+0_f(x)=f(x)+0_k=f(x)=f$
			\item Definiremos al inverso aditivo como la funci\'on $-f:X\to K$ tal que $\forall x\in X$ resulte que $f(x)+(-f(x))=0_k$, luego: \\
			$f+(-f)=(f+(-f))(x)=f(x)+(-f(x))=0_k=0_f$
			\item $
				\alpha(\beta f)=(\alpha(\beta f))(x)=\alpha(\beta f)(x)=\alpha \beta f(x)=(\alpha \beta)f(x)=((\alpha \beta)f)(x)=(\alpha \beta)f
				$
			\item $
				\alpha(f+g)=(\alpha(f+g))(x)=\alpha(f+g)(x)=\alpha(f(x)+g(x))=\alpha f(x)+\alpha g(x)=(\alpha f)(x)+(\alpha g)(x)=\alpha f+\alpha g
				$
			\item $
				(\alpha+\beta)f=((\alpha+\beta)f)(x)=(\alpha+\beta)f(x)=\alpha f(x)+\beta f(x)=(\alpha f)(x)+(\beta f)(x)=\alpha f+\beta f
				$
			\item $
				1_k\cdot f=(1_k\cdot f)(x)=1_k\cdot f(x)=f(x)=f
				$
		\end{enumerate}
	\end{solution}

	% PREGUNTA 3
	\question Sea $V=\mathbb{R}^2$. Sean $(x_1, x_2),(y_1, y_2)$. Definimos la una suma $\oplus$ como sigue $(x_1, x_2)\oplus (y_1, y_2) =(x_1 + 2y_1, x_2-y_2)$. ?`Es $V$ con esta suma y con el producto por escalares usual un espacio vectorial sobre $\mathbb{R}$? En caso de serlo, demuestrelo; de lo contrario, ejemplifique todas las propiedades que no cumple.
	\begin{solution}
		No es un espacio vectorial, ya que no cumple con la asociaci\'on y conmutaci\'on sobre $\oplus$:
		\begin{enumerate}[label=\roman*)]
			\item Sea $x=(2,4),y=(1,3),z=(5,6)$, tenemos que $x,y,z\in V$ pero $x\oplus (y\oplus z)=(2,4)\oplus (1+2\cdot 5,3-6)=(2,4)\oplus (11,-3)=(2+2\cdot 11,4+3)=(24,7)\neq(14,-5)=(4+2\cdot 5,1-6)=(4,1)\oplus (5,6)=(2+2\cdot 1, 4-3)\oplus (5,6)=(x\oplus y)\oplus z$.
			\item Sea $x=(1,2), y=(3,6)$, tenemos que $x,y\in V$, pero $x\oplus y=(1+2\cdot 3,2-6)=(7,-4)\neq (5,4)=(3+2\cdot 1,6-2)=y\oplus x$.
		\end{enumerate}
	\end{solution}

	% PREGUNTA 4
	\question Sea $V=\mathbb{R}^3$ el espacio vectorial sobre $\mathbb{R}$ con la suma y producto usual sobre $\mathbb{R}$. Determine si los siguientes subconjuntos son subespacios vectoriales de $V$. Justifique sus respuesta.
	\begin{parts}
		\part $W_1=\{(x,0,z)|x,z\in \mathbb{R}\}$
		\begin{solution}
			Sea $u=(u_1,0,u_2),v=(v_1,0,v_2)\in W_1$, sea $\alpha, \beta\in \mathbb{R}$. Luego $\alpha u+ \beta v=(\alpha u_1,0,\alpha u_2)+(\beta v_1,0,\beta v_2)=(\alpha u_1+\beta v_1,0+0,\alpha u_2+\beta v_2)=(\rho,0,\sigma)$, con $\rho = \alpha u_1+\beta v_1$ y $\sigma = \alpha u_2+\beta v_2$, y como $\rho,\sigma\in \mathbb{R}$ entonces $(\rho,0,\sigma)\in W_1$, por lo tanto es un subespacio vectorial.
		\end{solution}

		\part $W_2=\{(x,y,z)|(x,y,z)=t(2,1,1)-(0,1,1),t\in\mathbb{R}\}$
		\begin{solution}
			No es un subespacio vectorial, ya que $\neg\exists t\in\mathbb{R}$ tal que $t(2,1,1)-(0,1,1)=(2t,t-1,t-1)=(0,0,0)=0_v.$
		\end{solution}

		\part $W_3=\{(x,y,z)|x=2y=3z\}$
		\begin{solution}
				Sea $u=(a,b,c),v=(r,s,t)\in W_3$, por lo tanto podemos expresar a los vectores como $u=(3c,\frac{3}{2}c,c)$ y $v=(3t,\frac{3}{2}t,t)$. Sea $\alpha,\beta\in\mathbb{R}$ luego 
				$\alpha u+\beta v=
				\alpha(3c,\frac{3}{2}c,c)+\beta(3t,\frac{3}{2}t,t)=
				(3\alpha c,\frac{3}{2}\alpha c,\alpha c)+(3\beta t,\frac{3}{2}\beta t,\beta t)
				=(3(\alpha c+\beta t),\frac{3}{2}(\alpha c+\beta t),\alpha c+\beta t)
				=(3\rho,\frac{3}{2}\rho,\rho)
				$ con $\rho=\alpha c+\beta t$, pero vemos que $(3\rho,\frac{3}{2}\rho,\rho)\in W_3$, por lo tanto es un subespacio vectorial.
		\end{solution}

		\part $W_4=\{(x,y,z)|xyz\geq 0\}$
		\begin{solution}
			No es un subespacio vectorial. Sea $u=(x,y,z)\in W_4$, $x=y=z\neq 0$, por lo tando $xyz\geq 0$, tenemos que $-1\in \mathbb{R}$, pero $-1\cdot(x,y,z)=(-x,-y,-z)\not\in W_4$ ya que $(-x)(-y)(-z)=-xyz\ngeq 0$.
		\end{solution}

		\part $W_5=\{(x,y,z)|2x-3y+5z=0\}$
		\begin{solution}
			Sea $u=(a,b,c),v=(r,s,t)\in W_5$, por lo tanto $2a-3b+5c=0$ y $2r-3s+5t=0$.Sea $\alpha\in\mathbb{R}$, luego $\alpha u=\alpha(\alpha a,\alpha b,\alpha c)$, vemos que $2\alpha a-3\alpha b+5\alpha c=\alpha(2a-3b+5c)=0$ y $u+v=(a+r,b+s,c+t)$ y vemos que $2(a+r)-3(b+s)+5(c+t)=(2a-3b+5c)+(2r-3s+5t)=0$, por lo tanto es subespacio vectorial.
		\end{solution}

		\part $W_6=\{(x,y,z)|\sqrt{x^2+y^2+z^2}\leq 1\}$
		\begin{solution}
			Tenemos que $u=(0.5,0.5,0.5)\in W_6$, ya que $\sqrt{0.5^2+0.5^2+0.5^2}=0.75\leq 1$, Luego $3\in\mathbb{R}$, asi $3(0.5,0.5,0.5)=(3(0.5),3(0.5),3(0.5))$, pero 
			$\sqrt{(3(0.5))^2+(3(0.5))^2+(3(0.5))^2}=$\\
			$
			\sqrt{9(0.5)^2+9(0.5)^2+9(0.5)^2}=
			\sqrt{9(0.5^2+0.5^2+0.5^2)}=
			3\sqrt{0.5^2+0.5^2+0.5^2}=2.25\nleq 1
			$, por lo tanto no es un subespacio vectorial.
		\end{solution}
	\end{parts}

	\question Sea $V=R_2[x]$, es decir el conjunto de los polinomios con coeficientes en los reales con grado menor o igual a 2, el espacio vectorial sobre $\mathbb{R}$ con la suma y producto usual. Determine si los siguientes subconjuntos son subespacios vectoriales de $V$. Justifique sus respuesta.
	\begin{parts}
		\part $W_1=\{ax^2+bx+c|a\leq 0\}.$
		\part $W_2=\{ax^2+c|ac\geq 0\}.$
		\part $W_3=\{p(x)|p(−x)=p(x)\}.$
		\part $W_4=\{p(x)|p(−x)=−p(x)\}.$
		\part $Sea \alpha\in \mathbb{R}.$ Entonces $W_5=\{p(x)|p(\alpha)=0\}.$
		\part $W_6=\{ax^2+bx+c|a+b+c=0\}.$
	\end{parts}

	\question Sea $V=\mathcal{M}_{2\times 2}(\mathbb{R})$, el espacio vectorial sobre $\mathbb{R}$ con la suma y producto usual. Determine si los siguientes subconjuntos son subespacios vectoriales de $V$. Justifique sus respuesta.
	\begin{parts}
		\part $W_1 = \{A||A| = 0\}$.
			\begin{solution}
				Tenemos que $ \forall A [\lnot(|A| = 0)\Leftrightarrow ($A es invertible$)]$
			\end{solution}
		\part $W_2 = \{A|A^2=0_{2\times 2}\}$.
		\part $W_3 = \{\begin{pmatrix}a&b\\0&d\end{pmatrix}|a,b,d\in \mathbb{R}\}$.
		\part $W_4 = \{\begin{pmatrix}a&-2b\\3b+a&-4a+b\end{pmatrix}|a,b\in \mathbb{R}\}$.
		\part $W_5 = \{\begin{pmatrix}a&b\\c&abc\end{pmatrix}|a,b,c\in \mathbb{R}\}$.
	\end{parts}

	\question Determine si el siguiente conjunto de polinomios genera a $R_2[x]$. Justifique su respuesta.
	\begin{parts}
		\part $\{1,x^2\}$
		\begin{solution}
			No genera a $R_2[x]$. Sea $a_0+a_1x+a_2x^2\in R_2[x]$, luego $\exists\alpha,\beta$ tales que $\alpha+\beta x^2=a_0+a_1x+a_2x^2$ lo que implica que $\alpha=a_0$ y $\beta=a_2$, pero $a_1$ queda sin asignaci\'on, lo que implica $a_1=0$.
		\end{solution}

		\part $\{3,2x,-x^2\}$
		\begin{solution}
			Sea $a_0+a_1x+a_2x^2\in R_2[x]$. Luego $\exists\alpha,\beta,\gamma$ tales que $3\alpha+2\beta x-\gamma x^2=a_0+a_1x+a_2x^2$, por lo tanto $3\alpha=a_0$, $2\beta=a_1$ y $3-\gamma=a_0$, es decir, $\alpha=\frac{a_0}{3}$, $\beta=\frac{a_1}{2}$ y $\gamma=-a_0$, por lo tanto generan a $R_2[x]$.
		\end{solution}

		\part $\{1+x,2+2x-3x^2,x^2\}$
		\begin{solution}
			Sea $a_0+a_1x+a_2x^2\in R_2[x]$. Luego $\exists\alpha,\beta,\gamma$ tales que $\alpha(1+x)+\beta(2+2x-3x^2)+\gamma(x^2)=a_0+a_1x+a_2x^2$, asi:\\
			$\alpha(1+x)+\beta(2+2x-3x^2)+\gamma(x^2)=(\alpha+2\beta)+(\alpha+2\beta)x+(-3\beta+\gamma)x^2=a_0+a_1x+a_2x^2$\\
			Por lo tanto tenemos el siguiente sistema (cuyas indeterminadas son $\alpha,\beta$ y $\gamma$):\\
			$\begin{pmatrix}[ccc|c]
				1 & 2 & 0 & a_0\\
				1 & 2 & 0 & a_1\\
				0 & -3 & 1 & a_2\\
			\end{pmatrix}
			\rightarrow
			\begin{pmatrix}[ccc|c]
				1 & 2 & 0 & a_0\\
				0 & 0 & 0 & a_1-a_0\\
				0 & -3 & 1 & a_2\\
			\end{pmatrix}$\\
			por lo tanto nos restringimos a que $a_1-a_0=0$, es decir $a_1=a_0$, por lo que no genera a $R_2[x]$.
		\end{solution}

		\part $\{1,1+x,1+x^2\}$
		\begin{solution}
			Sea $a_0+a_1x+a_2x^2\in R_2[x]$. Luego $\exists\alpha,\beta,\gamma$ tales que $\alpha+\beta(1+x)+\gamma(1+x^2)=a_0+a_1x+a_2x^2$, asi:\\
			$\alpha+\beta(1+x)+\gamma(1+x^2)=(\alpha+\beta+\gamma)+\beta x+\gamma x^2=a_0+a_1x+a_2x^2$\\
			Por lo tanto tenemos el siguiente sistema (cuyas indeterminadas son $\alpha,\beta$ y $\gamma$):\\
			$\begin{pmatrix}[ccc|c]
				1 & 1 & 1 & a_0\\
				0 & 1 & 0 & a_1\\
				0 & 0 & 1 & a_2\\
			\end{pmatrix}
			\rightarrow
			\begin{pmatrix}[ccc|c]
				1 & 0 & 1 & a_0-a_1\\
				0 & 1 & 0 & a_1\\
				0 & 0 & 1 & a_2\\
			\end{pmatrix}
			\rightarrow
			\begin{pmatrix}[ccc|c]
				1 & 0 & 0 & a_0-a_1-a_2\\
				0 & 1 & 0 & a_1\\
				0 & 0 & 1 & a_2\\
			\end{pmatrix}
			$\\
			por lo tanto el sistema tiene solucion unica, as\'i el conjunto genera a $R_2[x]$.
		\end{solution}
	\end{parts}

	\question Para cada conjunto del inciso anterior, determine si son linealmente independientes.

	\question Determine que figuras geometricas representan los siguientes subespacios de $\mathbb{R}^3$.
	\begin{parts}
		\part $\langle\{\begin{pmatrix}2\\-1\\4\end{pmatrix},\begin{pmatrix}1\\4\\6\end{pmatrix}\}\rangle$
		\begin{solution}
			Representa un plano, ya que los vectores son no paralelos, por lo que solo se necesita calcular su vector normal para crear la ecuaci\'on del plano.
		\end{solution}

		\part $\langle\{\begin{pmatrix}-1\\3\\2\end{pmatrix},\begin{pmatrix}2\\-6\\-4\end{pmatrix}\}\rangle$
		\begin{solution}
			Representa una recta, ya que el segundo vector es igual a $-2$ veces el primer vector, por lo que uno puede actuar como el punto $P_0$ de la recta mientras que el otro puede actuar como el vector direcci\'on.
		\end{solution}
	\end{parts}

	\question Determine para que valores de $a$ y $b$ los vectores $\begin{pmatrix}1\\2\\a\\1\end{pmatrix}$, $\begin{pmatrix}1\\2\\3\\b\end{pmatrix}$, $\begin{pmatrix}0\\1\\b\\0\end{pmatrix}$ son $l.d.$ en $\mathbb{R}^4$.

	\question Determine si los polinomios $1+x$, $2+4x^2$, $4x+5x^3$, $2x^2-x^4$, $-2+2x^4$, $1-x+x^2-x^3+x^4$ son $l.i.$ o son $l.d.$ en $R_4[x]$.
	\begin{solution}
		Son $l.d.$, ya que $dim(R_4[x])=5$, por lo tanto un conjunto de 6 elementos (como el que se nos da) no puede ser $l.i.$
	\end{solution}

	\question Sean $v,u,w\in V$ , donde $V$ es un espacio vectorial. Pruebe que si $u$,$v$ y $w$ son linealmente independientes, entonces $u+v$, $u+w$ y $v+w$ son linealmente independientes.

	\question Encuentre una base para los siguientes subespacios de $\mathbb{R}^3$.
\end{questions}

\par Hecho en \LaTeX.
\end{document}
\begin{comment}
		\part $W_2=\{ax^2+c|ac\geq 0\}$
		\begin{solution}
			No es un espacio vectorial. Sea $u=2x^2+10,v=-10x^2-2$, $u,v\in W_2$ ya que $(2)(10)=20\geq 0$ y $(-10)(-2)=20\geq 0$ pero $u+v=(2x^2+10)+(-10x^2-2)=(2-10)x^2+(10-2)=-8x^2+8\not\in W_2$ ya que $(-8)(8)=-64\ngeq 0$.
		\end{solution}

		\part $W_3=\{p(x)|p(-x)=p(x)\}$
		\begin{solution}
			Sea $p(x),q(x)\in W_3$ Supongamos que $M=p(x)=p(-x)$ y $N=q(x)=q(-x)$. Sean $\alpha,\beta\in \mathbb{R}$, luego $\alpha p(x)+\beta q(x)=\alpha L +\beta M=\alpha p(-x)+\beta q(-x)$, por lo tanto $\alpha p(x)+\beta q(x)\in W_3$, as\'i $W_3$ es un subespacio vectorial.
		\end{solution}

		\part $W_4=\{p(x)|p(-x)=-p(x)\}$
		\begin{solution}
			Sea $p(x),q(x)\in W_4$ Supongamos que $M=p(x)$ y $N=q(x)$. Sean $\alpha,\beta\in \mathbb{R}$, luego $\alpha p(x)+\beta q(x)=\alpha L +\beta M=\alpha p(-x)+\beta q(-x)$, por lo tanto $\alpha p(x)+\beta q(x)\in W_3$, as\'i $W_3$ es un subespacio vectorial.
		\end{solution}
\end{comment}